%https://www.overleaf.com/learn/latex/code_listing
\definecolor{GREEN}{rgb}{0,0.6,0}
\definecolor{GRAY}{rgb}{0.5,0.5,0.5}
\definecolor{PURPLE}{rgb}{0.58,0,0.82}
\definecolor{BACKGROUND}{rgb}{0.95,0.95,0.92}

\lstdefinestyle{CODE}{
  backgroundcolor=\color{BACKGROUND},
  commentstyle=\color{GREEN},
  keywordstyle=\color{blue},
  numberstyle=\tiny\color{GRAY},
  stringstyle=\color{PURPLE},
  basicstyle=\ttfamily\footnotesize,
  language=C,
  tabsize=4,
  showspaces=false,
  showstringspaces=false,
  frame=single,
  breaklines=true,
  captionpos=b,
  postbreak=\mbox{\textcolor{red}{$\hookrightarrow$}\space},
}
\lstset{style=CODE}


\chapter{Compiler Extensions}

\begin{bclogo}[logo=\bctrombone, noborder=true, couleurBarre=blue!30]{Files}
  \small
  \begin{tabular}{l l}
      \faPlug & \emph{src/api.h} \\
      \faWrench & \emph{src/api.h} \\
      \faBalanceScale & \emph{test/compiler-extensions.c} \\
      \faBook & \emph{doc/chapters/compiler-extensions.tex} \\
  \end{tabular}
\end{bclogo}

\begin{bclogo}[logo=\bctakecare, noborder=true, couleurBarre=orange]{Warning}
  The compiler extension macros have no effect on compilers other than GCC and
  Clang.
\end{bclogo}

GCC provides a rich set of compiler extensions to the C language standards
through the \verb|__attribute__| keyword. These compiler extensions allow 
special properties to be specified to types, enumerators, statements, functions, 
and variables. These special properties, applied through the 
\verb|__attribute__| keyword, are used by GCC to perform optimisations and
provide other enhancements beyond what the C standard specifies.

Experience has shown that careful use of these extensions is beneficial,
especially if we know that the code is to be compiled on GCC, or a GCC-
compatible compiler such as Clang that supports the \verb|__attribute__| 
keyword.

The Argent Library uses a few selected such extensions, and wraps them as macros
so that they degrade gracefully (with an appropriate warning) when ported to 
compilers that do not support the \verb|__attribute__| extensions provided
by GCC.

The list of compiler extension macros are summarised in Table~\ref{tab:synopsis}
and elaborated in the following sections. These macros are declared in the
\emph{src/api.h} header file.

\renewcommand\arraystretch{1.1}
\begin{table}[!htbp]
\centering
\small
\begin{tabular}[t]{>{\centering}m{0.3\linewidth}
    >{\raggedright\arraybackslash}m{0.6\linewidth}}
\toprule
\textbf{Macro} & \textbf{Synopsis} \\
\midrule
\verb|ag_cold| & Hints that a given function is called rarely \\
\verb|ag_hot| & Hints that a given function is called frequently \\
\verb|ag_likely| & Hints that a predicate is likely to be true \\
\verb|ag_pure| & Hints that a given function is pure \\
\verb|ag_threadlocal| & Hints that a given variable has thread local 
    storage \\
\verb|ag_unlikely| & Hints that a predicate is likely to be false \\
\bottomrule
\end{tabular}
\caption{Synopsis of compiler extension macros}
\label{tab:synopsis}
\end{table}


\section{Macro \texttt{ag\_hot}}

\begin{bclogo}[logo=\bccrayon, noborder=true, barre=snake, couleurBarre=gray]
    {Synopsis}
  \verb|#include <argent/api.h>|
  \verb|#define ag_hot| \ldots
\end{bclogo}

A \emph{hot function} is one that is called frequently. Marking such hot
functions with the \verb|ag_hot| macro allows GCC to perform suitable
performance optimisations on them. To mark a function as hot, its prototype
needs to be decorated with \verb|ag_hot| like so:
\begin{lstlisting}[linewidth=1.0\linewidth]
extern ag_hot int foo(const char *bar);
\end{lstlisting}




\section{Macro \texttt{ag\_cold}}

\begin{bclogo}[logo=\bccrayon, noborder=true, barre=snake, couleurBarre=gray]
    {Synopsis}
  \verb|#include <argent/api.h>|
  \verb|#define ag_cold| \ldots
\end{bclogo}

The \verb|ag_cold| macro is similar to the \verb|ag_hot| macro, except
that it is used to mark a \emph{cold function}, i.e. a function which is called
rarely. To mark a function as cold, its prototype must be decorated like so:
\begin{lstlisting}[linewidth=1.0\linewidth]
extern ag_cold int foo(const char *bar);
\end{lstlisting}



\section{Macro \texttt{ag\_likely()}}

\begin{bclogo}[logo=\bccrayon, noborder=true, barre=snake, couleurBarre=gray]
    {Synopsis}
  \verb|#include <argent/api.h>|
  \verb|#define ag_likely(p)| \ldots
  \par
  \texttt{p} = \emph{predicate}
\end{bclogo}

The \verb|ag_likely()| macro provides a branch prediction hint that a predicate
is likely to be \textbf{true}. It has only one parameter, which is the predicate 
to be marked as likely. As an example, to mark the predicate of an \verb|if| 
condition as likely, we would write:
\begin{lstlisting}[linewidth=1.0\linewidth]
if (ag_likely (s && *s)) 
    printf("%s\n", s);
\end{lstlisting}



\section{Macro \texttt{ag\_unlikely()}}

\begin{bclogo}[logo=\bccrayon, noborder=true, barre=snake, couleurBarre=gray]
    {Synopsis}
  \verb|#include <argent/api.h>|
  \verb|#define ag_unlikely(p)| \ldots
  \par
  \texttt{p} = \emph{predicate}
\end{bclogo}

The \verb|ag_unlikely()| macro, similar to the \verb|ag_likely()| macro,
provides a branch prediction hint that a predicate is likely to be 
\textbf{false}. As an example, to mark the predicate of an \verb|if| condition 
as unlikely, we would write:

\begin{lstlisting}[linewidth=1.0\linewidth]
if (ag_unlikely (!(s && *s))) 
    printf("unable to display contents of 's'\n");
\end{lstlisting}


\section{Macro \texttt{ag\_pure}}

\begin{bclogo}[logo=\bccrayon, noborder=true, barre=snake, couleurBarre=gray]
    {Synopsis}
  \verb|#include <argent/api.h>|
  \verb|#define ag_pure| \ldots
\end{bclogo}

A \emph{pure function} is one that always returns the same result for the same
set of arguments without any side effects. In order for GCC to consider a
function to be pure, it must meet the following criteria:
\begin{enumerate}
  \item Its return value depends only on its arguments and global variables.
  \item It does not modify any of its arguments or global variables.
  \item It does not involve any I/O calls such as \texttt{printf()}.
  \item It does not make a call to any impure function.
\end{enumerate}

By marking such functions as pure with the \texttt{ag\_pure} macro, GCC is able
to perform relevant optimisations on them such as subexpression elimination. In
order to mark a function as pure, apply the \texttt{ag\_pure} macro on the
prototype declaration like so:

\begin{lstlisting}[linewidth=1.0\linewidth]
extern ag_pure int foo(const char *bar);
\end{lstlisting}

