%https://www.overleaf.com/learn/latex/code_listing
\definecolor{GREEN}{rgb}{0,0.6,0}
\definecolor{GRAY}{rgb}{0.5,0.5,0.5}
\definecolor{PURPLE}{rgb}{0.58,0,0.82}
\definecolor{BACKGROUND}{rgb}{0.95,0.95,0.92}

\lstdefinestyle{CODE}{
  backgroundcolor=\color{BACKGROUND},
  commentstyle=\color{GREEN},
  keywordstyle=\color{magenta},
  numberstyle=\tiny\color{GRAY},
  stringstyle=\color{PURPLE},
  basicstyle=\ttfamily\footnotesize,
  language=C,
  tabsize=4,
  showspaces=false,
  showstringspaces=false,
  frame=single,
  breaklines=true,
  captionpos=b,
  postbreak=\mbox{\textcolor{red}{$\hookrightarrow$}\space},
}
\lstset{style=CODE}


\chapter{Compiler Extensions}

GCC provides a rich set of compiler extensions to the C language standards
through the \texttt{\_\_attribute\_\_} keyword. These compiler extensions allow 
special properties to be specified to types, enumerators, statements, functions, 
and variables. These special properties, applied through the 
\texttt{\_\_attribute\_\_} keyword, are used by GCC to perform optimisations and
provide other enhancements beyond what the C standard specifies.

Experience has shown that careful use of these extensions is beneficial,
especially if we know that the code is to be compiled on GCC, or a GCC-
compatible compiler such as Clang that supports the 
\texttt{\_\_attribute\_\_} keyword.

The Argent Library uses a few selected such extensions, and wraps them as macros
so that they degrade gracefully (with an appropriate warning) when ported to 
compilers that do not support the \texttt{\_\_attribute\_\_} extensions provided
by GCC.

The list of compiler extension macros are summarised in Table~\ref{tab:synopsis}
and elaborated in the following sections. These macros are declared in the
\emph{src/api.h} header file.

\renewcommand\arraystretch{1.1}
\begin{table}[!htbp]
\centering
\small
\begin{tabular}[t]{>{\centering}m{0.3\linewidth}
    >{\raggedright\arraybackslash}m{0.6\linewidth}}
\toprule
\textbf{Macro} & \textbf{Synopsis} \\
\midrule
\texttt{ag\_cold} & Hints that a given function is called rarely \\
\texttt{ag\_hot} & Hints that a given function is called frequently \\
\texttt{ag\_likely} & Hints that a predicate is likely to be true \\
\texttt{ag\_pure} & Hints that a given function is pure \\
\texttt{ag\_threadlocal} & Hints that a given variable has thread local 
    storage \\
\texttt{ag\_unlikely} & Hints that a predicate is likely to be false \\
\bottomrule
\end{tabular}
\caption{Synopsis of compiler extension macros}
\label{tab:synopsis}
\end{table}


\section{Macro \texttt{ag\_hot}}
A \emph{hot function} is one that is called frequently. Marking such hot
functions with the \verb|ag_hot| macro allows GCC to perform suitable
performance optimisations on them. To mark a function as hot, its prototype
needs to be decorated with \verb|ag_hot| like so:
\begin{lstlisting}[linewidth=1.0\linewidth]
extern int ag_hot foo(const char *bar);
\end{lstlisting}

On compilers that do not support \verb|ag_hot|, the macro degrades safely
but generates a warning message during compilation.


\section{Macro \texttt{ag\_cold}}
The \verb|ag_cold| macro is similar to the \verb|ag_hot| macro, except
that it is used to mark a \emph{cold function}, i.e. a function which is called
rarely. To mark a function as cold, its prototype must be decorated like so:
\begin{lstlisting}[linewidth=1.0\linewidth]
extern int ag_cold foo(const char *bar);
\end{lstlisting}

If the compiler does not support \texttt{ag\_hot}, then the macro degrades
safely, generating an appropriate compiler warning.

\section{Macro \texttt{ag\_likely()}}
The \verb|ag_likely()| macro provides a branch prediction hint that a predicate
is likely to be \textbf{true}. It has only one parameter, which is the predicate 
to be marked as likely. As an example, to mark the predicate of an \verb|if| 
condition as likely, we would write:
\begin{lstlisting}[linewidth=1.0\linewidth]
if (ag_likely (s && *s)) 
    printf("%s\n", s);
\end{lstlisting}

On compilers that do not support \verb|ag_likely()|, the macro degrades safely
to a no-op, but issues an appropriate compiler warning.

\section{Macro \texttt{ag\_unlikely()}}
The \verb|ag_unlikely()| macro, similar to the \verb|ag_likely()| macro,
provides a branch prediction hint that a predicate is likely to be 
\textbf{false}. As an example, to mark the predicate of an \verb|if| condition 
as unlikely, we would write:
\begin{lstlisting}[linewidth=1.0\linewidth]
if (ag_unlikely (!(s && *s))) 
    printf("unable to display contents of 's'\n");
\end{lstlisting}

If the compiler does not support the \verb|ag_unlikely()| macro, then it
degrades saftely and emits a compiler warning.

\section{Macro \texttt{ag\_pure}}

