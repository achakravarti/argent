\chapter{Compiler Extensions}

GCC provides a rich set of compiler extensions to the C language standards
through the \texttt{\_\_attribute\_\_} keyword. These compiler extensions allow 
special properties to be specified to types, enumerators, statements, functions, 
and variables. These special properties, applied through the 
\texttt{\_\_attribute\_\_} keyword, are used by GCC to perform optimisations and
provide other enhancements beyond what the C standard specifies.

Experience has shown that careful use of these extensions is beneficial,
especially if we know that the code is to be compiled on GCC, or a GCC-
compatible compiler such as Clang that supports the 
\texttt{\_\_attribute\_\_} keyword.

The Argent Library uses a few selected such extensions, and wraps them as macros
so that they degrade gracefully (with an appropriate warning) when ported to 
compilers that do not support the \texttt{\_\_attribute\_\_} extensions provided
by GCC.

The list of compiler extension macros are summarised in Table~\ref{tab:synopsis}
and elaborated in the following sections. These macros are declared in the
\emph{src/api.h} header file.

\renewcommand\arraystretch{1.1}
\begin{table}[!htbp]
\centering
\small
\begin{tabular}[t]{>{\centering}m{0.3\linewidth}
    >{\raggedright\arraybackslash}m{0.6\linewidth}}
\toprule
\textbf{Macro} & \textbf{Synopsis} \\
\midrule
\texttt{ag\_cold} & Hints that a given function is called rarely \\
\texttt{ag\_hot} & Hints that a given function is called frequently \\
\texttt{ag\_likely} & Hints that a predicate is likely to be true \\
\texttt{ag\_pure} & Hints that a given function is pure \\
\texttt{ag\_threadlocal} & Hints that a given variable has thread local 
    storage \\
\texttt{ag\_unlikely} & Hints that a predicate is likely to be false \\
\bottomrule
\end{tabular}
\caption{Synopsis of compiler extension macros}
\label{tab:synopsis}
\end{table}


\section{Macro \texttt{ag\_hot}}
A \emph{hot function} is one that is called frequently. Marking such hot
functions with the \texttt{ag\_hot} macro allows GCC to perform suitable
performance optimisations on them. To mark a function as hot, its prototype
needs to be decorated with \texttt{ag\_hot} like so:
\par
\texttt{extern int ag\_hot foo(const char *bar);}

On compilers that do not support \texttt{ag\_hot}, the macro degrades safely
but generates a warning message during compilation.


\section{Macro \texttt{ag\_cold}}
The \texttt{ag\_cold} macro is similar to the \texttt{ag\_hot} macro, except
that it is used to mark a \emph{cold function}, i.e. a function which is called
rarely. To mark a function as cold, its prototype must be decorated like so:
\par
\texttt{extern int ag\_cold foo(const char *bar);}

If the compiler does not support \texttt{ag\_hot}, then the macro degrades
safely, generating an appropriate compiler warning.

\section{Macro \texttt{ag\_likely}}
\section{Macro \texttt{ag\_unlikely}}
\section{Macro \texttt{ag\_pure}}

