%https://www.overleaf.com/learn/latex/code_listing
\definecolor{GREEN}{rgb}{0,0.6,0}
\definecolor{GRAY}{rgb}{0.5,0.5,0.5}
\definecolor{PURPLE}{rgb}{0.58,0,0.82}
\definecolor{BACKGROUND}{rgb}{0.95,0.95,0.92}

\lstdefinestyle{CODE}{
  backgroundcolor=\color{BACKGROUND},
  commentstyle=\color{GREEN},
  keywordstyle=\color{blue},
  numberstyle=\tiny\color{GRAY},
  stringstyle=\color{PURPLE},
  basicstyle=\ttfamily\scriptsize,
  language=C,
  tabsize=4,
  showspaces=false,
  showstringspaces=false,
  frame=single,
  breaklines=true,
  captionpos=b,
  postbreak=\mbox{\textcolor{red}{$\hookrightarrow$}\space},
}
\lstset{style=CODE}


\chapter{Compiler Extensions}

\begin{bclogo}[logo=\bctrombone, noborder=true, couleurBarre=blue!30]{Files}
  \small
  \begin{tabular}{l l}
    \faPlug & \verb|src/api.h| \\
    \faWrench & \verb|src/api.h| \\
    \faBalanceScale & \verb|test/compiler-extensions.c| \\
    \faBook & \verb|doc/chp/compiler-extensions.tex| \\
  \end{tabular}
\end{bclogo}

\begin{bclogo}[logo=\bctakecare, noborder=true, couleurBarre=orange]{Warning}
  \small
  The compiler extension macros have no effect on compilers other than GCC and
  Clang.
\end{bclogo}

GCC provides a rich set of compiler extensions to the C language standards
through non-standard keywords and built-in functions such as 
\verb|__attribute__|, \verb|__thread| and \verb|__builtin_expect()|. These 
compiler extensions allow special properties to be specified to types, 
enumerators, statements, functions and variables. These special properties are 
used by GCC to perform optimisations and provide other enhancements beyond what 
the C standard specifies.

Experience has shown that careful use of these extensions is beneficial,
especially if we know that the code is to be compiled on GCC, or a GCC-
compatible compiler such as Clang that supports the non-standard keywords and
built-in functions introduced by GCC.

The Argent Library uses a few selected such extensions, and wraps them as macros
so that they degrade gracefully (with an appropriate warning) when ported to 
compilers that do not support the non-standard extensions provided by GCC.

The list of compiler extension macros are summarised in Table~\ref{tab:synopsis}
and elaborated in the following sections. These macros are both declared and
defined in the \verb|src/api.h| header file.

\renewcommand\arraystretch{1.1}
\begin{table}[!htbp]
  \small
  \centering
  \caption{Synopsis of compiler extension macros}
  \label{tab:synopsis}
  \begin{tabular}[t]{>{\centering}m{0.3\linewidth}
    >{\raggedright\arraybackslash}m{0.6\linewidth}}
    \toprule
    \textbf{Macro} & \textbf{Synopsis} \\
    \midrule
    \verb|ag_cold| & Hints that a given function is called rarely \\
    \verb|ag_hot| & Hints that a given function is called frequently \\
    \verb|ag_likely| & Hints that a predicate is likely to be true \\
    \verb|ag_pure| & Hints that a given function is pure \\
    \verb|ag_threadlocal| & Hints that a given variable has thread local 
      storage \\
    \verb|ag_unlikely| & Hints that a predicate is likely to be false \\
    \bottomrule
  \end{tabular}
\end{table}


%%%%%%%%%%%%%%%%%%%%%%%%%%%%%%%%%%%%%%%%%%%%%%%%%%%%%%%%%%%%%%%%%%%%%%%%%%%%%%%%
% Macro ag_pure
%


\section{Macro \texttt{ag\_pure}}

  \begin{bclogo}[logo=\bccrayon, noborder=true, barre=snake, couleurBarre=gray]
    {Synopsis}
    \small
    \verb|#include <argent/api.h>| \\
    \verb|#define ag_pure| \ldots
  \end{bclogo}

  A \emph{pure function} is one that always returns the same result for the same
  set of arguments without any side effects. In order for GCC to consider a
  function to be pure, it must meet the following criteria:
  \begin{enumerate}
    \item Its return value depends only on its arguments and global variables.
    \item It does not modify any of its arguments or global variables.
    \item It does not involve any I/O calls such as \texttt{printf()}.
    \item It does not make a call to any impure function.
  \end{enumerate}

  By marking such functions as pure with the \texttt{ag\_pure} macro, GCC is able
  to perform relevant optimisations on them such as subexpression elimination. In
  order to mark a function as pure, apply the \texttt{ag\_pure} macro on the
  prototype declaration like so:

  \begin{lstlisting}[linewidth=1.0\linewidth, caption=Example use of ag\_pure]
    extern ag_pure int foo(const char *bar);
  \end{lstlisting}

  The algorithm for defining \verb|ag_pure| is given by Algorithm
  \ref{alg:ag_pure}. We first check whether the compiler is either GCC or Clang;
  if either is detected, then \verb|ag_pure| is defined as
  \verb|__attribute__((pure))|, otherwise it is defined to nothing and an
  appropriate compiler warning is issued.

  \begin{algorithm}
    \scriptsize
    \caption{Defining \texttt{ag\_pure}}
    \label{alg:ag_pure}
  \begin{algorithmic}
    \If{$compiler \equiv gcc$ or $compiler \equiv clang$}
      \State{$\verb|ag_pure| \gets \verb|__attribute__((pure))|$}
    \Else
      \State{$\verb|ag_pure| \gets \emptyset$}
      \Comment{disable ag\_pure}
      \State{$\verb|#warning| \gets$ ``unsupported macro''}
      \Comment{compiler warning}
    \EndIf
  \end{algorithmic}
  \end{algorithm}


%%%%%%%%%%%%%%%%%%%%%%%%%%%%%%%%%%%%%%%%%%%%%%%%%%%%%%%%%%%%%%%%%%%%%%%%%%%%%%%%
% Macro ag_hot
%


\section{Macro \texttt{ag\_hot}}

\begin{bclogo}[logo=\bccrayon, noborder=true, barre=snake, couleurBarre=gray]
  {Synopsis}
  \small
  \verb|#include <argent/api.h>| \\
  \verb|#define ag_hot| \ldots
\end{bclogo}

A \emph{hot function} is one that is called frequently. Marking such hot
functions with the \verb|ag_hot| macro allows GCC to perform suitable
performance optimisations on them. To mark a function as hot, its prototype
needs to be decorated with \verb|ag_hot| like so:

\begin{lstlisting}[linewidth=1.0\linewidth, 
    caption=Example use of ag\_hot]
extern int foo(const char *bar) ag_hot;
\end{lstlisting}

The algorithm for defining \verb|ag_hot| is given by Algorithm \ref{alg:ag_hot}.
This algorithm is essentially the same as that for defining \verb|ag_pure|
(Algorithm \ref{alg:ag_pure}), except for the use of \verb|__attribute__((hot))|
instead of \verb|__attribute__((pure))|.

  \begin{algorithm}
    \scriptsize
    \caption{Defining \texttt{ag\_hot}}
    \label{alg:ag_hot}
  \begin{algorithmic}
    \If{$compiler \equiv gcc$ or $compiler \equiv clang$}
      \State{$\verb|ag_hot| \gets \verb|__attribute__((hot))|$}
    \Else
      \State{$\verb|ag_hot| \gets \emptyset$}
      \Comment{disable ag\_hot}
      \State{$\verb|#warning| \gets$ ``unsupported macro''}
      \Comment{compiler warning}
    \EndIf
  \end{algorithmic}
  \end{algorithm}

%%%%%%%%%%%%%%%%%%%%%%%%%%%%%%%%%%%%%%%%%%%%%%%%%%%%%%%%%%%%%%%%%%%%%%%%%%%%%%%%
% Macro ag_cold
%


\section{Macro \texttt{ag\_cold}}

\begin{bclogo}[logo=\bccrayon, noborder=true, barre=snake, couleurBarre=gray]
  {Synopsis}
  \small
  \verb|#include <argent/api.h>| \\
  \verb|#define ag_cold| \ldots
\end{bclogo}

The \verb|ag_cold| macro is similar to the \verb|ag_hot| macro, except
that it is used to mark a \emph{cold function}, i.e. a function which is called
rarely. To mark a function as cold, its prototype must be decorated like so:

\begin{lstlisting}[linewidth=1.0\linewidth,
    caption=Example use of ag\_cold]
extern int foo(const char *bar) ag_cold;
\end{lstlisting}

The algorithm for defining \verb|ag_cold| is given by Algorithm 
\ref{alg:ag_cold}. Just as in the case of the algorithm for defining 
\verb|ag_hot| (Algorithm \ref{alg:ag_hot}, this algorithm is essentially the 
same as that for defining \verb|ag_pure| (Algorithm \ref{alg:ag_pure}), except 
for the use of \verb|__attribute__((cold))| instead of 
\verb|__attribute__((pure))|.

  \begin{algorithm}
    \scriptsize
    \caption{Defining \texttt{ag\_cold}}
    \label{alg:ag_cold}
  \begin{algorithmic}
    \If{$compiler \equiv gcc$ or $compiler \equiv clang$}
      \State{$\verb|ag_cold| \gets \verb|__attribute__((cold))|$}
    \Else
      \State{$\verb|ag_cold| \gets \emptyset$}
      \Comment{disable ag\_cold}
      \State{$\verb|#warning| \gets$ ``unsupported macro''}
      \Comment{compiler warning}
    \EndIf
  \end{algorithmic}
  \end{algorithm}


%%%%%%%%%%%%%%%%%%%%%%%%%%%%%%%%%%%%%%%%%%%%%%%%%%%%%%%%%%%%%%%%%%%%%%%%%%%%%%%%
% Macro ag_likely()
%


\section{Macro \texttt{ag\_likely()}}

\begin{bclogo}[logo=\bccrayon, noborder=true, barre=snake, couleurBarre=gray]
  {Synopsis}
  \small
  \verb|#include <argent/api.h>| \\
  \verb|#define ag_likely(p)| \ldots \par
  \texttt{p} = \emph{predicate to hint}
\end{bclogo}

The \verb|ag_likely()| macro provides a branch prediction hint that a predicate
is likely to be \textbf{true}. It has only one parameter, which is the predicate 
to be marked as likely. As an example, to mark the predicate of an \verb|if| 
condition as likely, we would write:

\begin{lstlisting}[linewidth=1.0\linewidth,
    caption=Example use of ag\_likely()]
if (ag_likely (s && *s)) 
    printf("%s\n", s);
\end{lstlisting}

The algorithm for defining the \verb|ag_likely()| macro is given by Algorithm
\ref{alg:ag_likely}, and has been modelled after the definition of the
\verb|likely()| macro in the Linux kernel. 

We first check whether GCC or Clang is being used as the compiler; if so, then 
the \verb|ag_likely()| macro is defined in terms of the GCC built-in function 
\verb|__builtin_expect()|, passing the predicate \emph{p} to be compared against 
the value \textbf{1}. The double negation of the predicate \emph{p} ensures that 
it evaluates to either \textbf{0} or \textbf{1}. If we find that GCC or Clang is
not being used, then we simply alias \verb|ag_likely()| to the predicate
\emph{p} and issue an appropriate compiler warning.

  \begin{algorithm}
    \scriptsize
    \caption{Defining \texttt{ag\_likely()}}
    \label{alg:ag_likely}
  \begin{algorithmic}
    \If{$compiler \equiv gcc$ or $compiler \equiv clang$}
      \State{$\verb|ag_likely(p)| \gets \Call{$\_\_builtin\_expect$}{!!(p), 1}$}
    \Else
      \State{$\verb|ag_likely(p)| \gets (p)$}
      \Comment{disable ag\_likely()}
      \State{$\verb|#warning| \gets$ ``unsupported macro''}
      \Comment{compiler warning}
    \EndIf
  \end{algorithmic}
  \end{algorithm}


%%%%%%%%%%%%%%%%%%%%%%%%%%%%%%%%%%%%%%%%%%%%%%%%%%%%%%%%%%%%%%%%%%%%%%%%%%%%%%%%
% Macro ag_unlikely()
%


\section{Macro \texttt{ag\_unlikely()}}

\begin{bclogo}[logo=\bccrayon, noborder=true, barre=snake, couleurBarre=gray]
  {Synopsis}
  \small
  \verb|#include <argent/api.h>| \\
  \verb|#define ag_unlikely(p)| \ldots \par
  \texttt{p} = \emph{predicate to hint}
\end{bclogo}

The \verb|ag_unlikely()| macro, similar to the \verb|ag_likely()| macro,
provides a branch prediction hint that a predicate is likely to be 
\textbf{false}. As an example, to mark the predicate of an \verb|if| condition 
as unlikely, we would write:

\begin{lstlisting}[linewidth=1.0\linewidth,
    caption=Example use of ag\_unlikely()]
if (ag_unlikely (!(s && *s))) 
    printf("unable to show contents of 's'\n");
\end{lstlisting}
 
 The algorithm for the definition of \verb|ag_unlikely()| is given by Algorithm
 \ref{alg:ag_unlikely}, and is modelled after the definition of the
 \verb|unlikely()| macro in the Linux kernel. This algorithm is similar to that
 of \verb|ag_likely()| (Algorithm \ref{alg:ag_likely}), except that we compare
 the predicate against \textbf{0} in the call to \verb|__builtin_expect()|.

  \begin{algorithm}
    \scriptsize
    \caption{Defining \texttt{ag\_unlikely()}}
    \label{alg:ag_unlikely}
  \begin{algorithmic}
    \If{$compiler \equiv gcc$ or $compiler \equiv clang$}
      \State{$\verb|ag_unlikely(p)| \gets \Call{$\_\_builtin\_expect$}{!!(p), 
      0}$}
    \Else
      \State{$\verb|ag_unlikely(p)| \gets (p)$}
      \Comment{disable ag\_unlikely()}
      \State{$\verb|#warning| \gets$ ``unsupported macro''}
      \Comment{compiler warning}
    \EndIf
  \end{algorithmic}
  \end{algorithm}


%%%%%%%%%%%%%%%%%%%%%%%%%%%%%%%%%%%%%%%%%%%%%%%%%%%%%%%%%%%%%%%%%%%%%%%%%%%%%%%%
% Macro ag_threadlocal
%


\section{Macro \texttt{ag\_threadlocal}}

\begin{bclogo}[logo=\bccrayon, noborder=true, barre=snake, couleurBarre=gray]
  {Synopsis}
  \small
  \verb|#include <argent/api.h>| \\
  \verb|#define ag_threadlocal| \ldots
\end{bclogo}

Thread local storage refers to the mechanism by which a global or static
variable is given as a separate copy to each thread, thereby keeping the 
variable local to the thread.

\begin{lstlisting}[linewidth=1.0\linewidth,
    caption=Example use of ag\_threadlocal]
  static ag_threadlocal int foo = 55;
\end{lstlisting}

The algorithm for defining the \verb|ag_threadlocal| macro is given by Algorithm
\ref{alg:ag_threadlocal}. We start by checking the C dialect being used; if we
are using the C11 (and beyond) dialect, we include the standard \emph{threads.h} 
header file and define \verb|ag_threadlocal| to be an alias for
the standard \verb|thread_local| keyword.

If we are using a dialect lower than C11, we check if we are using either GCC or
Clang as the compiler; if so, then we define \verb|ag_threadlocal| to be an 
alias for the GCC-specific keyword \verb|__thread|, otherwise we define
\verb|ag_threadlocal| to nothing and issue an appropriate compiler warning.
  \begin{algorithm}
    \scriptsize
    \caption{Defining \texttt{ag\_threadlocal}}
    \label{alg:ag_threadlocal}
  \begin{algorithmic}
    \If{$STDC \geq C11$}
      \State{$\verb|#include| \gets \verb|<threads.h>|$}
      \State{$\verb|ag_threadlocal| \gets \verb|thread_local|$}
    \ElsIf{$compiler \equiv gcc$ or $compiler \equiv clang$}
      \State{$\verb|ag_threadlocal| \gets \verb|__thread|$}
    \Else
      \State{$\verb|ag_threadlocal| \gets \emptyset$}
      \Comment{disable ag\_threadlocal}
      \State{$\verb|#warning| \gets$ ``unsupported macro''}
      \Comment{compiler warning}
    \EndIf
  \end{algorithmic}
  \end{algorithm}
