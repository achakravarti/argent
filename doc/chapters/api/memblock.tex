%https://www.overleaf.com/learn/latex/code_listing
\definecolor{GREEN}{rgb}{0,0.6,0}
\definecolor{GRAY}{rgb}{0.5,0.5,0.5}
\definecolor{PURPLE}{rgb}{0.58,0,0.82}
\definecolor{BACKGROUND}{rgb}{0.95,0.95,0.92}

\lstdefinestyle{CODE}{
  backgroundcolor=\color{BACKGROUND},
  commentstyle=\color{GREEN},
  keywordstyle=\color{blue},
  numberstyle=\tiny\color{GRAY},
  stringstyle=\color{PURPLE},
  basicstyle=\ttfamily\scriptsize,
  language=C,
  tabsize=4,
  showspaces=false,
  showstringspaces=false,
  frame=single,
  breaklines=true,
  captionpos=b,
  postbreak=\mbox{\textcolor{red}{$\hookrightarrow$}\space},
}

\lstdefinestyle{SYNOPSIS}{
  backgroundcolor=\color{white},
  keywordstyle=\color{blue},
  basicstyle=\ttfamily\scriptsize,
  language=C,
  directivestyle=\color{blue},
  frame=none,
  breaklines=true,
  postbreak=\mbox{\textcolor{red}{$\hookrightarrow$}\space}
}

\newenvironment{WrapText}[1][r]
  {\wrapfigure{#1}{0.5\textwidth}\tcolorbox}
  {\endtcolorbox\endwrapfigure}

\chapter{Memory Block Module}

\begin{bclogo}[logo=\bctrombone, noborder=true, couleurBarre=blue!30]{Files}
\small
\begin{tabular}{l l}
\faPlug & \verb|src/api.h| \\
\faWrench & \verb|src/api.h| \\
\faBalanceScale & \verb|test/memblock.c| \\
\faBook & \verb|doc/chp/memblock.tex| \\
\end{tabular}
\end{bclogo}

\renewcommand\arraystretch{1.1}
\begin{table}[!htbp]
  \small
  \centering
  \caption{Synopsis of functions}
  \label{tab:synopsis-memblock-functions}
  \begin{tabular}[t]{>{\centering}m{0.4\linewidth}
    >{\raggedright\arraybackslash}m{0.5\linewidth}}
    \toprule
    \textbf{Function} & \textbf{Synopsis} \\
    \midrule
    \verb|ag_memblock_new()| & Allocates a new memory block \\
    \verb|ag_memblock_copy()| & Copies a memory block \\
    \verb|ag_memblock_sz()| & Gets the size of a memory block \\
    \verb|ag_memblock_resize()| & Resizes a memory block \\
    \verb|ag_memblock_free()| & Releases a memory block \\
    \bottomrule
  \end{tabular}
\end{table}


%%%%%%%%%%%%%%%%%%%%%%%%%%%%%%%%%%%%%%%%%%%%%%%%%%%%%%%%%%%%%%%%%%%%%%%%%%%%%%%%
%		      Section: Function ag_memblock_new()
%

\section{Function \texttt{ag\_memblock\_new()}}

\begin{noteblock}
\small
\verb|ag_memblock_new()|---allocates block of heap memory; returns allocated
block of heap memory.
\lstset{style=SYNOPSIS}
\begin{lstlisting}[linewidth=1.0\linewidth]
#include <argent/api.h>
extern ag_memblock_t *ag_memblock_new(size_t sz);
\end{lstlisting}
\verb|sz| = size in bytes to allocate.
\end{noteblock}

The algorithm for the implementation of \verb|ag_memblock_new()| is given by
Algorithm \ref{alg:ag_memblock_new}.

\begin{WrapText}

\begin{algorithm}
\caption{\texttt{ag\_memblock\_new()}}
\label{alg:ag_memblock_new}
\begin{algorithmic}
\Require $sz \geq 0$
\Procedure{ag\_memblock\_new}{$sz$}
\State $bfr \gets malloc(sz)$
\State $bfr \gets memset(0)$
\State return $bfr$
\EndProcedure
\Ensure $bfr \neq null$
\end{algorithmic}
\end{algorithm}

\end{WrapText}

\lstset{style=CODE}
\lstinputlisting[linerange={35-43},caption=Definition of ag\_memblock\_new()]
  {../src/memblock.c}

%%%%%%%%%%%%%%%%%%%%%%%%%%%%%%%%%%%%%%%%%%%%%%%%%%%%%%%%%%%%%%%%%%%%%%%%%%%%%%%%
%		      Section: Function ag_memblock_copy()
%


\section{Function \texttt{ag\_memblock\_copy()}}

\begin{noteblock}
\small
\verb|ag_memblock_copy()|---copies existing block of heap memory; returns copied
block of heap memory.
\lstset{style=SYNOPSIS}
\begin{lstlisting}[linewidth=1.0\linewidth]
#include <argent/api.h>
extern ag_memblock_t *ag_memblock_copy(const ag_memblock_t *bfr);
\end{lstlisting}
\verb|bfr| = heap block to copy.
\end{noteblock}

The algorithm for the implementation of \verb|ag_memblock_copy()| is given by
Algorithm \ref{alg:ag_memblock_copy}.

\begin{algorithm}
\scriptsize
\caption{\texttt{ag\_memblock\_copy()}}
\label{alg:ag_memblock_copy}
\begin{algorithmic}
\Require $bfr \neq null$
\Procedure{ag\_memblock\_copy}{$bfr$}
\State $sz \gets ag\_memblock\_sz(bfr)$ \Comment{compute size of bfr}
\State $cp \gets ag\_memblock\_new(sz)$ \Comment{create new block of size sz}
\State $cp \gets memcpy(bfr)$ \Comment{copy bfr to cp}
\State return $cp$
\EndProcedure
\end{algorithmic}
\end{algorithm}

\lstset{style=CODE}
\lstinputlisting[linerange={47-56},caption=Definition of ag\_memblock\_copy()]
  {../src/memblock.c}


%%%%%%%%%%%%%%%%%%%%%%%%%%%%%%%%%%%%%%%%%%%%%%%%%%%%%%%%%%%%%%%%%%%%%%%%%%%%%%%%
%		       Section: Function ag_memblock_sz()
%


\section{Function \texttt{ag\_memblock\_sz()}}

\begin{noteblock}
\small
\verb|ag_memblock_sz()|---gets size block of heap memory; returns size in bytes
of heap block.
\lstset{style=SYNOPSIS}
\begin{lstlisting}[linewidth=1.0\linewidth]
#include <argent/api.h>
extern size_t ag_memblock_sz(const ag_memblock_t *bfr);
\end{lstlisting}
\verb|bfr| = heap block to query.
\end{noteblock}

The algorithm for the implementation of \verb|ag_memblock_sz()| is given by
Algorithm \ref{alg:ag_memblock_sz}.

\begin{algorithm}
\scriptsize
\caption{\texttt{ag\_memblock\_sz()}}
\label{alg:ag_memblock_sz}
\begin{algorithmic}
\Require $bfr \neq null$
\Procedure{ag\_memblock\_copy}{$bfr$}
\State $sz \gets ag\_memblock\_sz(bfr)$ \Comment{compute size of bfr}
\State $cp \gets ag\_memblock\_new(sz)$ \Comment{create new block of size sz}
\State $cp \gets memcpy(bfr)$ \Comment{copy bfr to cp}
\State return $cp$
\EndProcedure
\end{algorithmic}
\end{algorithm}


\lstset{style=CODE}
\lstinputlisting[linerange={60-64},caption=Definition of ag\_memblock\_sz()]
  {../src/memblock.c}


%%%%%%%%%%%%%%%%%%%%%%%%%%%%%%%%%%%%%%%%%%%%%%%%%%%%%%%%%%%%%%%%%%%%%%%%%%%%%%%%
%		     Section: Function ag_memblock_resize()
%


\section{Function \texttt{ag\_memblock\_resize()}}

\begin{noteblock}
\small
\verb|ag_memblock_resize()|---resizes allocated block of heap memory.
\lstset{style=SYNOPSIS}
\begin{lstlisting}[linewidth=1.0\linewidth]
#include <argent/api.h>
extern void ag_memblock_resize(ag_memblock_t **bfr, size_t sz);
\end{lstlisting}
\verb|bfr| = heap block to resize, \par
\verb|sz| = new size in bytes.
\end{noteblock}

\lstset{style=CODE}
\lstinputlisting[linerange={68-73},caption=Definition of ag\_memblock\_resize()]
  {../src/memblock.c}



%%%%%%%%%%%%%%%%%%%%%%%%%%%%%%%%%%%%%%%%%%%%%%%%%%%%%%%%%%%%%%%%%%%%%%%%%%%%%%%%
%		      Section: Function ag_memblock_free()
%


\section{Function \texttt{ag\_memblock\_free()}}

\begin{noteblock}
\small
\verb|ag_memblock_free()|---releases allocated block of heap memory
\lstset{style=SYNOPSIS}
\begin{lstlisting}
#include <argent/api.h>
extern void ag_memblock_free(ag_memblock_t **bfr);
\end{lstlisting}
\verb|bfr| = heap block to release.
\end{noteblock}

\lstset{style=CODE}
\lstinputlisting[linerange={77-83},caption=Definition of ag\_memblock\_resize()]
  {../src/memblock.c}
