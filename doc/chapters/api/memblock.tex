%https://www.overleaf.com/learn/latex/code_listing
\definecolor{GREEN}{rgb}{0,0.6,0}
\definecolor{GRAY}{rgb}{0.5,0.5,0.5}
\definecolor{PURPLE}{rgb}{0.58,0,0.82}
\definecolor{BACKGROUND}{rgb}{0.95,0.95,0.92}

\lstdefinestyle{CODE}{
  backgroundcolor=\color{BACKGROUND},
  commentstyle=\color{GREEN},
  keywordstyle=\color{blue},
  numberstyle=\tiny\color{GRAY},
  stringstyle=\color{PURPLE},
  basicstyle=\ttfamily\scriptsize,
  language=C,
  tabsize=4,
  showspaces=false,
  showstringspaces=false,
  frame=single,
  breaklines=true,
  captionpos=b,
  postbreak=\mbox{\textcolor{red}{$\hookrightarrow$}\space},
}

\lstdefinestyle{SYNOPSIS}{
  backgroundcolor=\color{white},
  keywordstyle=\color{blue},
  basicstyle=\ttfamily\scriptsize,
  language=C,
  directivestyle=\color{blue},
  frame=none,
  breaklines=true,
  postbreak=\mbox{\textcolor{red}{$\hookrightarrow$}\space}
}

\chapter{Memory Block Module}

\begin{bclogo}[logo=\bctrombone, noborder=true, couleurBarre=blue!30]{Files}
  \small
  \begin{tabular}{l l}
    \faPlug & \verb|src/api.h| \\
    \faWrench & \verb|src/api.h| \\
    \faBalanceScale & \verb|test/memblock.c| \\
    \faBook & \verb|doc/chp/memblock.tex| \\
  \end{tabular}
\end{bclogo}

\renewcommand\arraystretch{1.1}
\begin{table}[!htbp]
  \small
  \centering
  \caption{Synopsis of functions}
  \label{tab:synopsis-memblock-functions}
  \begin{tabular}[t]{>{\centering}m{0.4\linewidth}
    >{\raggedright\arraybackslash}m{0.5\linewidth}}
    \toprule
    \textbf{Function} & \textbf{Synopsis} \\
    \midrule
    \verb|ag_memblock_new()| & Allocates a new memory block \\
    \verb|ag_memblock_copy()| & Copies a memory block \\
    \verb|ag_memblock_sz()| & Gets the size of a memory block \\
    \verb|ag_memblock_resize()| & Resizes a memory block \\
    \verb|ag_memblock_free()| & Releases a memory block \\
    \bottomrule
  \end{tabular}
\end{table}


%%%%%%%%%%%%%%%%%%%%%%%%%%%%%%%%%%%%%%%%%%%%%%%%%%%%%%%%%%%%%%%%%%%%%%%%%%%%%%%%
%		      Section: Function ag_memblock_new()
%

\section{Function \texttt{ag\_memblock\_new()}}
  \begin{bclogo}[logo=\bccrayon, noborder=true, barre=snake, couleurBarre=gray]
    {Synopsis}
  \lstset{style=SYNOPSIS}
  \begin{lstlisting}[linewidth=1.0\linewidth]
#include <argent/api.h>
extern ag_memblock_t *ag_memblock_new(size_t sz);
  \end{lstlisting}
  \scriptsize
  \texttt{sz} = \emph{allocation size in bytes}
  \end{bclogo}

\lstset{style=CODE}
\begin{lstlisting}[linewidth=1.0\linewidth,
    caption=Definition of ag\_memblock\_new()]
extern ag_memblock_t *ag_memblock_new(size_t sz)
{
    ag_assert (sz);
    void *bfr = malloc(sz);
    ag_require (bfr, AG_ERNO_MEMBLOCK_NEW, NULL);

    memset(bfr, 0, sz);
    return bfr;
}
\end{lstlisting}


%%%%%%%%%%%%%%%%%%%%%%%%%%%%%%%%%%%%%%%%%%%%%%%%%%%%%%%%%%%%%%%%%%%%%%%%%%%%%%%%
%		      Section: Function ag_memblock_copy()
%


\section{Function \texttt{ag\_memblock\_copy()}}
  \begin{bclogo}[logo=\bccrayon, noborder=true, barre=snake, couleurBarre=gray]
    {Synopsis}
  \lstset{style=SYNOPSIS}
  \begin{lstlisting}[linewidth=1.0\linewidth]
#include <argent/api.h>
extern ag_memblock_t *ag_memblock_copy(const ag_memblock_t *bfr);
  \end{lstlisting}
  \scriptsize
  \texttt{bfr} = \emph{buffer to copy}
  \end{bclogo}

\lstset{style=CODE}
\begin{lstlisting}[linewidth=1.0\linewidth,
    caption=Definition of ag\_memblock\_copy()]
extern ag_memblock_t *ag_memblock_copy(const ag_memblock_t *bfr)
{
    ag_assert (bfr);
    size_t sz = ag_memblock_sz(bfr);

    ag_memblock_t *cp = ag_memblock_new(sz);
    memcpy(cp, bfr, sz);

    return cp;
}
\end{lstlisting}


%%%%%%%%%%%%%%%%%%%%%%%%%%%%%%%%%%%%%%%%%%%%%%%%%%%%%%%%%%%%%%%%%%%%%%%%%%%%%%%%
%		       Section: Function ag_memblock_sz()
%


\section{Function \texttt{ag\_memblock\_sz()}}
  \begin{bclogo}[logo=\bccrayon, noborder=true, barre=snake, couleurBarre=gray]
    {Synopsis}
  \lstset{style=SYNOPSIS}
  \begin{lstlisting}[linewidth=1.0\linewidth]
#include <argent/api.h>
extern size_t ag_memblock_sz(const ag_memblock_t *bfr);
  \end{lstlisting}
  \scriptsize
  \texttt{bfr} = \emph{buffer for which to determine size}
  \end{bclogo}

\lstset{style=CODE}
\begin{lstlisting}[linewidth=1.0\linewidth,
    caption=Definition of ag\_memblock\_sz()]
extern size_t ag_memblock_sz(const ag_memblock_t *bfr)
{
    ag_assert (bfr);
    return malloc_usable_size((void *) bfr);
}
\end{lstlisting}


%%%%%%%%%%%%%%%%%%%%%%%%%%%%%%%%%%%%%%%%%%%%%%%%%%%%%%%%%%%%%%%%%%%%%%%%%%%%%%%%
%		     Section: Function ag_memblock_resize()
%


\section{Function \texttt{ag\_memblock\_resize()}}
  \begin{bclogo}[logo=\bccrayon, noborder=true, barre=snake, couleurBarre=gray]
    {Synopsis}
  \lstset{style=SYNOPSIS}
  \begin{lstlisting}[linewidth=1.0\linewidth]
#include <argent/api.h>
extern void ag_memblock_resize(ag_memblock_t **bfr, size_t sz);
  \end{lstlisting}
  \scriptsize
  \texttt{bfr} = \emph{buffer to resize} \\
  \texttt{sz} = \emph{new size in bytes}
  \end{bclogo}

\lstset{style=CODE}
\begin{lstlisting}[linewidth=1.0\linewidth,
    caption=Definition of ag\_memblock\_resize()]
extern void ag_memblock_resize(ag_memblock_t **bfr, size_t sz)
{
    ag_assert (bfr && *bfr && sz);
    *bfr = realloc(*bfr, sz);
    ag_require (*bfr, AG_ERNO_MEMBLOCK_RESIZE, NULL);
}
\end{lstlisting}


%%%%%%%%%%%%%%%%%%%%%%%%%%%%%%%%%%%%%%%%%%%%%%%%%%%%%%%%%%%%%%%%%%%%%%%%%%%%%%%%
%		      Section: Function ag_memblock_free()
%


\section{Function \texttt{ag\_memblock\_free()}}
  \begin{bclogo}[logo=\bccrayon, noborder=true, barre=snake, couleurBarre=gray]
    {Synopsis}
  \lstset{style=SYNOPSIS}
  \begin{lstlisting}[linewidth=1.0\linewidth]
#include <argent/api.h>
extern void ag_memblock_free(ag_memblock_t **bfr);
  \end{lstlisting}
  \scriptsize
  \texttt{bfr} = \emph{buffer to release} \\
  \end{bclogo}

\lstset{style=CODE}
\begin{lstlisting}[linewidth=1.0\linewidth,
    caption=Definition of ag\_memblock\_free()]
extern void ag_memblock_free(ag_memblock_t **bfr)
{
    if (ag_likely (bfr && *bfr)) {
        free(*bfr);
        *bfr = NULL;
    }
}

\end{lstlisting}

