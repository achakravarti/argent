%
% document colours
%
\definecolor{pastelblue}{RGB}{68,78,81}
\definecolor{green}{rgb}{0,0.6,0}
\definecolor{gray}{rgb}{0.5,0.5,0.5}
\definecolor{purple}{rgb}{0.58,0,0.82}
\definecolor{silver}{rgb}{0.75,0.75,0.75}


%
% code declaration list style
%
\lstdefinestyle{declaration}{
  backgroundcolor=\color{white},
  basicstyle=\ttfamily\scriptsize,
  breaklines=true,
  captionpos=b,
  commentstyle=\color{green},
  frame=none,
  keywordstyle=\color{blue},
  language=C,
  numberstyle=\tiny\color{green},
  postbreak=\mbox{\textcolor{green}{$\hookrightarrow$}\space},
  showspaces=false,
  showstringspaces=false,
  stringstyle=\color{orange},
  tabsize=4,
}


%
% code definition list style
%
\lstdefinestyle{definition}{
  backgroundcolor=\color{silver},
  basicstyle=\ttfamily\scriptsize,
  breaklines=true,
  captionpos=b,
  commentstyle=\color{green},
  frame=single,
  keywordstyle=\color{blue},
  language=C,
  numberstyle=\tiny\color{green},
  postbreak=\mbox{\textcolor{green}{$\hookrightarrow$}\space},
  showspaces=false,
  showstringspaces=false,
  stringstyle=\color{orange},
  tabsize=4,
}

%
% command aliases for ag_memblock_t interface functions
%
\newcommand{\agMemblockNew}{\texttt{ag\_memblock\_new()}}
\newcommand{\agMemblockCopy}{\texttt{ag\_memblock\_copy()}}
\newcommand{\agMemblockSz}{\texttt{ag\_memblock\_sz()}}
\newcommand{\agMemblockResize}{\texttt{ag\_memblock\_resize()}}
\newcommand{\agMemblockFree}{\texttt{ag\_memblock\_free()}}


%
% command to show function synopsis
% \agFnSynopsis{description}{linerange}{parameters}
%
\newcommand{\agFnSynopsis}[3]{
  \begin{wrapfigure}{l}{0.6\textwidth}
  \begin{minipage}{0.6\textwidth}
  \vspace{-1em}
  \begin{noteblock}
  \small
  #1.
  \lstset{style=declaration}
  \lstinputlisting[linerange={#2}]{../src/api.h}
  #3.
  \end{noteblock}
  \end{minipage}
  \end{wrapfigure}
}


%
% command to show show algorithm
% \agAlgorithm{label}{caption}{algorithm}
%
\newcommand{\agAlgorithm}[3]{
  \begin{wrapfigure}{r}{.6\textwidth}
  \begin{minipage}{0.6\textwidth}
  \vspace{-1em}
  \begin{algorithm}[H]
  \small
  \caption{#2}
  \label{#1}
  \begin{algorithmic}
  #3
  \end{algorithmic}
  \end{algorithm}
  \end{minipage}
  \end{wrapfigure}
}


\chapter{Memory Block Module}
\ThisULCornerWallPaper{0.5}{kirsty-paget.jpg}
\newpage


%
% section ag_memblock_new()
%
\section{Function \agMemblockNew}
\agFnSynopsis{\agMemblockNew---allocates block of heap memory; returns allocated
  block of heap memory}{265-265}{\texttt{sz} = size in bytes to allocate}

\agMemblockNew  is responsible for allocating a new block of heap memory. The 
size in bytes to allocate is passed through the \verb|sz| parameter. The value 
of \verb|sz| is required to be greater than zero; this condition is asserted in 
debug builds.

If successful, a handle to the new block of heap memory is returned. In case of 
failure to allocate heap memory, \agMemblockNew raises an exception with the 
\verb|AG_MEMBLOCK_NEW| error code, which is left for the exception handler 
callback function (defined by client code) to process. Hence, there is no need 
to check for the returned handle being non-\verb|NULL| in the normal 
flow of code.

It is important to note that the actual size of the allocated block may well be
greater than that specified by \verb|sz|. This is because the underlying
\verb|malloc()| function may allocate a larger block in order to account for
alignment and minimum size constraints.

\agAlgorithm{alg:agMemblockNew}{\agMemblockNew}{
  \Require $sz > 0$
  \Procedure{ag\_memblock\_new}{$sz$}
  \State $bfr \gets malloc(sz)$
  \State $bfr \gets memset(0)$
  \State return $bfr$
  \EndProcedure
  \Ensure $bfr \neq null$
}

The two most likely reasons for failure are the value of \verb|sz| being
larger than the amount of heap memory physically available, and excess
fragmentation of the heap. Neither of these failure conditions have been tested,
though.

The implementation of \agMemblockNew is quite straight-forward, essentially
being a wrapper around the standard \verb|malloc()| function along with an
in-built check for success. The algorithm for \agMemblockNew is given by 
Algorithm \ref{alg:agMemblockNew}.

We start off by asserting that the size of allocation specified through the
\verb|sz| parameter is $>$ 0, and then call \verb|malloc()| to allocated the
required amount of heap memory. We ensure that the handle to the allocated block
is valid; if not, we raise the \verb|AG_ERNO_MEMBLOCK_NEW| exception, passing
control to the client-provided exception handler callback without any extra
data. If the alloction has been successful, then we initialise the allocated
memory to zero before returning the handle to it.


%
% section ag_memblock_copy()
%
\section{Function \agMemblockCopy}
\agFnSynopsis{\agMemblockCopy---copies existing block of heap memory; returns
  copied block of heap memory}{274-274}{\texttt{bfr} = heap block to copy}

\agMemblockCopy makes a deep copy of an existing block of heap memory, passed as
a handle to its only parameter \verb|bfr|. \verb|bfr| is required to be valid;
this is asserted in debug builds. Additionally, it is required for \verb|bfr| to
have been allocated on the heap by an earlier call to \agMemblockNew (or 
\agMemblockCopy itself); passing a handle to a block of memory on the stack is
an error, leading to undefined behaviour (most likely a segmentation fault).

\agAlgorithm{alg:agMemblockCopy}{\agMemblockCopy}{
  \Require $bfr \neq null$
  \Procedure{ag\_memblock\_copy}{$bfr$}
  \State $sz \gets ag\_memblock\_sz(bfr)$
  \State $cp \gets ag\_memblock\_new(sz)$ 
  \State $cp \gets memcpy(bfr)$ 
  \State return $cp$
  \EndProcedure
}

On successful completion, \agMemblockCopy \\ returns a handle to the newly 
allocated copy of \verb|bfr|. In case of failure to do so, the 
\verb|AG_ERNO_MEMBLOCK_NEW| exception is raised, and control is passed to the 
client-provided exception handler. Since in the normal flow of control the newly
allocated copy is guaranteed to be valid, there is no need for client code to 
make a further check on the return value.

Since \agMemblockCopy relies on a non-standard \verb|libc| call, it is 
important to remember the portability restrictions this entails; see section
\ref{sec:agMemblockSz} on page \pageref{sec:agMemblockSz} for more details.

The algorithm for \agMemblockCopy is given by Algorithm
\ref{alg:agMemblockCopy}. We start by asserting that the \verb|bfr| argument is
valid and computing its allocated size through a call to
\verb|ag_memblock_sz()|. We then create a new a new block of heap memory of the
same size, copy the contents of \verb|bfr| to this newly allocated block, and
then return the handle to this new block.


%
% section ag_memblock_sz()
%
\section{Function \agMemblockSz}
\label{sec:agMemblockSz}

\agFnSynopsis{\agMemblockSz---gets size of block of heap memory; returns size in
  bytes of heap block}{283-283}{\texttt{bfr} = heap block to query}

Sometimes it is useful to be determine the size of a block of heap memory that
was allocated earlier without having to resort to storing the size in a separate
variable. The interface function \agMemblockSz provides a way to do so, allowing
client code to query the size of a block of heap memory \verb|bfr| that was 
allocated by \agMemblockNew or \agMemblockCopy. 

\agMemblockSz requires that \verb|bfr| be a valid handle to a block of heap 
memory; this function asserts in debug builds that the handle is not 
\verb|NULL|. Passing a pointer to the stack memory is an error, and will result 
in undefined behaviour, most likely being a segmentation fault.

As in the case of \agMemblockNew, it is important to remember that the size
returned by \agMemblockSz may be greater than that which had been originally
requested at the time of allocation, since the underlying \verb|malloc()| call
needs to take into account alignment and size restrictions.

\agAlgorithm{alg:agMemblockSz}{\agMemblockSz}{
  \Require $bfr \neq null$
  \Procedure{ag\_memblock\_sz}{$bfr$}
  \State return $malloc\_usable\_size(bfr)$
  \EndProcedure
}

There is an important caveat regarding portability. The C standard does not
define a way to query the size of the heap block returned by \verb|malloc()|,
and so we need to rely on the functionality provided by the underlying
\verb|libc| implementation. As of the current version, \agMemblockSz is
guaranteed to work on GNU/Linux, Cygwin and the various BSDs.

The algorithm for \agMemblockSz is given by Algorithm \ref{alg:agMemblockSz}. We
assert that the block of heap memory to query is valid and pass it to
\verb|malloc_usable_size()|, returning the value thereby obtained.
\verb|malloc_usable_size()| is a non-standard function supported by both
\verb|glibc| and BSD \verb|libc|.


%
% section ag_memblock_resize()
%
\section{Function \agMemblockResize}
\label{sec:agMemblockResize}

\agFnSynopsis{\agMemblockResize---resizes allocated block of heap memory}
  {292-292}{\texttt{bfr} = heap block to resize,\par\texttt{sz} = new size in
  bytes}

<<<<<<< HEAD
The \agMemblockResize \\ function is responsible for resizing the amount of 
memory allocated to an existing block of heap memory. Although this function is 
not used commonly within the Argent Library itself, it may prove to be useful 
for client code, especially for the resizing of dynamic data structures that can
grow and shrink, such as vectors.

\agMemblockResize takes two arguments, the first being the handle to the heap
buffer \texttt{bfr} that is to be allocated, and the second being the new size 
\texttt{sz} value. \texttt{bfr} is expected to be a valid pointer to a pointer 
to a memory block, and \texttt{sz} is expected to be greater than zero; these 
two conditions are asserted in debug builds.

\agAlgorithm{alg:agMemblockResize}{\agMemblockResize}{
  \Require $bfr \neq null$
  \Require $*bfr \neq null$
  \Require $sz > 0$
  \Procedure{ag\_memblock\_resize}{$bfr, sz$}
  \State $*bfr \gets realloc(*bfr, sz)$
  \EndProcedure
  \Ensure $*bfr \neq null$
}

If successful, the heap memory block pointed to by \texttt{bfr} will be resized 
to the value specified by \texttt{sz}. Note that, as in the case of 
\agMemblockNew, the actual size of the heap buffer may be larger than that
requested in order to account for alignment and minimum size constraints. In
case the resizing operation failed, the \texttt{AG\_MEMBLOCK\_RESIZE} exception 
is raised. Since \agMemblockResize is guaranteed to either succeed, or to raise
an exception, there is no need to test whether the heap buffer pointed to by 
\texttt{bfr} is valid after the operation.

The algorithm for \agMemblockResize is given by Algorithm 
\ref{alg:agMemblockResize}. We assert that both the pointer to the heap memory 
buffer and the heap buffer itself are valid, and that the new size requested is 
greater than zero. We then call the standard C function \texttt{realloc()} to 
resize the heap buffer to the requested size, and subsequently ensure that the 
resizing operation has been successful.

%
% section ag_memblock_free()
%
\section{Function \agMemblockFree}
\agFnSynopsis{\agMemblockFree---releases allocated block of heap memory}
  {300-300}{\texttt{bfr} = heap block to release}
\kant[1]
